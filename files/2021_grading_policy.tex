\documentclass{article}
\usepackage[utf8]{inputenc}
\setlength{\parindent}{0cm}
\setlength{\parskip}{1ex}
\usepackage{hyperref}

\title{CSCI1410 Grading Policy}
\author{}
\date{Fall 2021}
\begin{document}

\maketitle

\section{Coding}
\subsection{Scoring a Submission}
The coding portions of the assignments will be autograded on Gradescope. Please follow the Gradescope Setup Guide on our website for setting up an anonymous Gradescope account. \\

There are two types of test suites: visible and hidden. Results against the visible test cases are shown to students immediately after all submits/resubmits. Results against the hidden test cases are shown after a certain date. See ``Submission Policy" for more details. \\

Your score will be computed according to a rubric that we will distribute to you with each assignment.
Below is an excerpt from an example rubric:

\begin{verbatim}
    ---bfs [20]
    --------produces correct solution paths
    ----states are expanded in a breadth-first order     [10]
\end{verbatim}

Each rubric contains \textbf{categories} (e.g., ``bfs'') and atomic \textbf{items} (e.g., ``produces correct solution paths'' and ``states are expanded in a breadth-first order'').
You can recognize an item as having no subcategories and its score total aligned with the score totals of other items. ~\\

For each item, \textbf{a submission will earn either full points, if it passes all of the tests associated with that item, or no points, if it fails any of the tests.}
Its total score is the sum of its scores on the items. 

\subsection{Submission Attempts}

Before the final deadline of each project (with the exception of the first install project), there will be only two chances to receive results from the hidden tests. These are optional, but highly recommended. You may resubmit as many times as you would like before each attempt date but you will only receive feedback from the visible tests. The hidden results will be displayed immediately after the attempt deadlines have past.  

There will be separate assignments created on Gradescope for the three submissions (2 attempts and the final submission).

For each of the projects, you grade will be the maximum score of the two attempts and the final submission. 

\subsubsection{Late Policy}
In addition to the submission policy, you are granted three late days. There will be a specific Gradescope assignment for late submissions that will be created for each project. You will only be able to see feedback from the hidden tests once the deadline for late submissions have passed (three days after the due date of the assignment). The number of late days used will be determined by date and time of your latest submission.

Your mark will be the maximum of the four submissions (2 attempts, the final submission, and the late submission). 

The intention of these late days is to give you flexibility to do things like attend interviews, or deal with scheduling conflicts, so please use them wisely.

If you are seriously ill, experience a tragedy, or have another good reason for which you will not be able to complete your work on time, you can email Professor Konidaris to request a further extension.
Only very serious cases will be considered.

\subsection{Coding Grade Complaints}
If you are suspicious that the autograder has not graded your code properly, follow this procedure:
\begin{enumerate}
    \item Verify with a TA, either at hours or over Piazza, that you understand both the ostensible reason for which you lost points and the specifications of that part of the assignment.
    
    \item Write unit tests that show your code achieves the desired functionality.
    
    \item Contact the HTAs via email. 
    
\end{enumerate}

\section{Written Work}
The written work that you submit will be graded by hand.
It is usually due at the 11:59pm of the day after the final due date of the assignment. 
Written work is collected through a separate assignment Gradescope.

\subsection{Written Work Anonymity}
We grade anonymously for the purpose of fairness.
To that end, please do not put your name, CS login, or banner ID on your written work.

\subsection{Written Work Resubmission and Lateness}
Gradescope will allow you to resubmit written work up to the deadline.
Only the latest submission of your written work will be scored. 
If you have used a late day towards your code submission, your written work is due 11:59pm of the next day.

\subsection{Written Work Grade Complaints}
To contest a grade, use the regrade request feature of Gradescope.
If you're having trouble receiving a regrade, contact the HTAs.

\section{Grade Breakdown}
The following is a tentative breakdown of how your final course grade will be calculated.
Note that this is subject to change at the discretion of the professor.

\begin{verbatim}
Projects
    Install Assignment ......1%
    Search...................10%
    Adversarial Search.......12%
    KRR......................10%
    HMM......................12%
    Reinforcement Learning...10%
    Supervised Learning......10%
Final Project: 25%
\end{verbatim}
\end{document}